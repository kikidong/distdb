\documentclass[12pt,a4papper]{report}
\usepackage{type1cm}
\usepackage[cm-default]{fontspec}
\usepackage{xunicode}
\usepackage[english]{babel}
\XeTeXlinebreaklocale='zh'
\XeTeXlinebreakskip = 0pt plus 1pt minus 0.1pt

\newfontinstance\CODEFONT{"Inconsolata"}
\setmainfont{宋体}
\setsansfont{黑体}

\newcommand{\code}[1]{  {\noindent\CODEFONT \mbox{#1}}  }

\begin{document}
%\tableofcontents

\chapter*{前言}
%\chaptername{前言}

\begin{quote}
所有的高级操作都是由最简单的原语实现的\\
\begin{flushright}
{\CODEFONT \mbox{---microcai}}
\end{flushright}
\end{quote}

数据库的原语就是各自在使用 socket 通讯的时候使用的协议,所有数据库的高级接
口(odbc啊 occi 啊)都是利用的这些低阶的原语实现的。

而这些原语,每个数据库都不一样。

如果 \code{distdb} 使用这些原语,那么,显然工作量和开发 occi 接口一样巨大。而这么巨大的工作量,只能支持\code{oracle}
数据库。如果要支持别的数据库,就要加倍的开发量

所以,找到一个所有数据库都能支持的,相对又比较低阶的原语就很重要。

目前有啊,很方便的就是 SQL 语句。已经很高级了,但是相对于项目本身,直接使用 SQL 语句操作数据库又是很底层的
操作。

而且所有的数据库都能支持一些 SQL 语句。这些共同支持的SQL 操作被 \code{distdb} 用来当作自己的原语。

对数据库的操作都可以被转换为 SQL 语句。对于不能转换的,\code{distdb} 无能为力。因为这是 特定于特定的数据库类型的


distdb 不再以独立程序运行,而是一个库。它在一个库所能做的范围之内搞定




\end{document}





 

